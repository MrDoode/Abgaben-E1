\section{Chemikalienliste}
\subsection{Kaliumpermanganat}
\begin{itemize}
\item H272 Kann Brand verstärken; Oxidationsmittel.\\
\item H302 Gesundheitsschädlich bei Verschlucken.\\
\item H314 Verursacht schwere Verätzungen der Haut und schwere Augenschäden.\\
\item H361d Kann vermutlich das Kind im Mutterleib schädigen.\\
\item H373 Kann die Organe schädigen bei längerer oder wiederholter Exposition.
\item H410 Sehr giftig für Wasserorganismen mit langfristiger Wirkung.\\
\item P210 Von Hitze, heißen Oberflächen, Funken, offenen Flammen sowie anderen Zündquellenarten fernhalten. Nicht rauchen.\\
\item P220 Von Kleidung und anderen brennbaren Materialien fernhalten. \\
\item P280 Schutzhandschuhe / Schutzkleidung / Augenschutz / Gesichtsschutz tragen.\\
\item P301+P330+P331 Bei Verschlucken: Mund ausspülen. Kein Erbrechen herbeiführen. 
\item P303+P361+P353 Bei Berührung mit der Haut [oder dem Haar]: Alle kontaminierten Kleidungsstücke sofort ausziehen. Haut mit Wasser abwaschen [oder duschen]. 
\item P305+P351+P338 Bei Kontakt mit den Augen: Einige Minuten lang behutsam mit Wasser spülen. Eventuell vorhandene Kontaktlinsen nach Möglichkeit entfernen. Weiter spülen.
\item P310 Sofort Giftinformationszentrum, oder Arzt anrufen.\\
\end{itemize}

\subsection{Kaliumnitrat}
\begin{itemize}
\item H272 Kann Brand verstärken; Oxidationsmittel.\\
\item P210 Von Hitze, heißen Oberflächen, Funken, offenen Flammen sowie anderen Zündquellenarten fernhalten. Nicht rauchen.\\
\end{itemize}
