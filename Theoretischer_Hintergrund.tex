\section{Theoretischer Hintergrung}
Die Überführungszahl beschreibt den Bruchteil des gesamten Elektrischen Stromes, der von einer bestimmten Sorte Ionen transportiert wird. In einer Lösung mit nur zwei Ionenarten kann man die Überführungszahl mit $I_+/I$ berechnen. $I$ ist dabei der Gesamtstrom und $I_+$ der durch die Kationen transpotierte Strom.
Der Quotient der gesuchten Konzentrationsänderungen lässt sich über die Gleichung %1
$\frac{1}{t_-}-1$ überführen.
$$\frac{\Delta c_{Anodenraum}}{\Delta c_{Kathodenraum}} = \frac{t_+}{t_-} = \frac{1-t_-}{t_-} = \frac{1}{t_-}- 1$$
Bei der Elektrolyse wird nur das Hydroxion umgewandelt, während das Kaliumion unverändert in der Lösung verweilt. Daraus folgt: %2
$$\left|\Delta c_{Anodeenraum}\right| = \left|\Delta c_{Kathodenraum}\right|$$
Laut Definition ist die Ladung Q: %3
$$ Q= I\cdot t$$
Für die Überführungszahl gilt $t_+$: %4
$$t_+ = \frac{I_+}{I} = \frac{Q_+}{Q}$$
Das erste Faradaysche Gesetz bildet folgenden Zusammenhang: %5
$$Q_+ = \Delta n \cdot z \cdot F$$
Weiter folgt: %6
 $$t_+ = \frac{\Delta c_{Kathode}\cdot V_{Kathode}\cdot F \cdot  z_{Kation}}{Q} = - \frac{\Delta c_{Anode} \cdot V_{Anode} \cdot F \cdot z_{Kation}}{Q} $$
Für die Berechnung der Überführungszahl der Hydroxidionen gilt: %7
$$ t_+ + t_- = 1 $$
Alternativ erhalten wir die Überführungszahl aus einer Visuellen bestimmung.
In dem zweiten Versuchsaufbau wird eine Höhendifferenz ermittelt ($\Delta t$).Die Höhendifferenz ($\Delta t$) geteilt durch die Zeit ($t$) ergibt die Wandergeschwindigkeit: %8
$$V_{MnO4^-} = \frac{\Delta h}{t}$$
Mit dem Bezug auf das Elektrischesfeld wird die Ionenbeweglichkeit $U$ erhalten %9
$$ u_{MnO4^-} = \frac{u_{MnO4^-}}{\vec{E}} = \frac{u_{MnO4^-} \cdot l}{u}$$
Die molare Leitfähigkeit ist proportional zu der Ionenbeweglichkeit über: %10
$$\lambda_{MnO4^-} = zF\cdot u_{MnO4^-}$$
Für die Leidfähigkeit $\kappa$ der gesamten Lösung gilt: %11
$$\kappa = \frac{I}{U}\cdot \frac{l}{A}$$
Für die Überführungszahl $t$ gilt: %12
 $$ t_{MnO4^-} = \frac{\lambda_{MnO4^-} \cdot c_{MnO4^-}}{\kappa}$$
